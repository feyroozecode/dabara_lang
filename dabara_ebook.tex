\documentclass[11pt,a4paper]{article}
\usepackage[utf8]{inputenc}
\usepackage[french]{babel}
\usepackage[T1]{fontenc}
\usepackage{lmodern}
\usepackage{geometry}
\usepackage{listings}
\usepackage{xcolor}
\usepackage{fancyhdr}
\usepackage{graphicx}
\usepackage{hyperref}
\usepackage{multicol}
\usepackage{enumitem}
\usepackage{tikz}
\usepackage{booktabs}
\usepackage{caption}

% Configuration de la page
\geometry{
    a4paper,
    left=2.5cm,
    right=2.5cm,
    top=2.5cm,
    bottom=2.5cm
}

% Configuration du code
\definecolor{codeblue}{rgb}{0.25,0.5,0.5}
\definecolor{codegray}{rgb}{0.5,0.5,0.5}
\definecolor{codepurple}{rgb}{0.58,0,0.82}
\definecolor{backcolour}{rgb}{0.95,0.95,0.92}

\lstdefinestyle{dabarastyle}{
    backgroundcolor=\color{backcolour},
    commentstyle=\color{codegray},
    keywordstyle=\color{codeblue},
    numberstyle=\tiny\color{codegray},
    stringstyle=\color{codepurple},
    basicstyle=\ttfamily\small,
    breakatwhitespace=false,
    breaklines=true,
    captionpos=b,
    keepspaces=true,
    numbers=left,
    numbersep=5pt,
    showspaces=false,
    showstringspaces=false,
    showtabs=false,
    tabsize=2,
    frame=single,
    frameround=tttt
}

\lstset{style=dabarastyle}

% En-têtes et pieds de page
\pagestyle{fancy}
\fancyhf{}
\fancyhead[L]{Dabara : Guide Rapide}
\fancyhead[R]{Programmation en Hausa}
\fancyfoot[C]{\thepage}

% Titre du document
\title{
    \Huge \textbf{Dabara} \\
    \Large Guide Rapide de Programmation en Hausa \\
    \vspace{0.5cm}
    \large Version 1.0
}
\author{}
\date{}

\begin{document}

% Page de titre
\maketitle
\thispagestyle{empty}

\vfill

\begin{center}
\textbf{À propos de ce guide}

Ce guide vous accompagne dans l'apprentissage du langage de programmation Dabara, conçu pour rendre la programmation accessible aux francophones et aux locuteurs de langues nigériennes.

\vspace{1cm}

\textbf{Public cible} \\
Débutants en programmation \\
Développeurs curieux du multilinguisme technique \\
Éducateurs en informatique

\vspace{1cm}

\textbf{Prérequis} \\
Aucun - ce guide commence depuis zéro \\
Un ordinateur avec accès à Internet \\
Désir d'apprendre et de créer

\vspace{1cm}

\textbf{Durée estimée} \\
Lecture complète : 4-6 heures \\
Avec exercices : 8-12 heures

\end{center}

\newpage

% Table des matières
\tableofcontents
\newpage

% Début du contenu
\section*{Introduction : Bienvenue dans Dabara}

Dabara est un langage de programmation innovant qui rompt avec la tradition anglo-saxonne dominante dans le monde de l'informatique. Inspiré par le haoussa, l'une des langues les plus parlées en Afrique, Dabara offre une alternative multiculturelle à l'apprentissage de la programmation.

\subsection*{Pourquoi Dabara ?}

\begin{itemize}[leftmargin=*]
    \item \textbf{Accessibilité linguistique} : Programmation en haoussa et français
    \item \textbf{Culture locale} : Respect des traditions linguistiques africaines
    \item \textbf{Apprentissage intuitif} : Syntaxe proche du langage naturel
    \item \textbf{Innovation pédagogique} : Nouvelle approche de l'enseignement
\end{itemize}

\subsection*{Ce que vous allez apprendre}

Ce guide couvre les bases essentielles de la programmation avec Dabara :

\begin{enumerate}
    \item Installation et configuration
    \item Variables et types de données
    \item Opérations mathématiques
    \item Structures conditionnelles
    \item Fonctions et modularité
    \item Interaction avec l'utilisateur
    \item Collections et listes
    \item Projets pratiques
\end{enumerate}

\newpage

\section{Installation et Configuration}

Avant de commencer à programmer, configurons votre environnement de développement.

\subsection{Méthodes d'installation}

\subsubsection{Méthode 1 : Via Cargo (Recommandée)}

Si vous avez Rust installé :

\begin{lstlisting}[language=bash]
# Installer Dabara depuis le dépôt officiel
cargo install --git https://github.com/votre-compte/dabara

# Vérifier l'installation
dabara --version
\end{lstlisting}

\subsubsection{Méthode 2 : Compilation depuis les sources}

\begin{lstlisting}[language=bash]
# Cloner le dépôt
git clone https://github.com/votre-compte/dabara
cd dabara

# Compiler
cargo build --release

# Copier l'exécutable dans votre PATH
sudo cp target/release/dabara /usr/local/bin/
\end{lstlisting}

\subsubsection{Méthode 3 : Téléchargement direct}

Téléchargez l'exécutable pré-compilé depuis la page des releases GitHub.

\subsection{Vérification de l'installation}

Testons que tout fonctionne correctement :

\begin{lstlisting}[language=bash]
# Créer un fichier test
echo 'fara
  rubuta "Barka da zuwa Dabara!"
kare' > test.ha

# Exécuter le programme
dabara test.ha
\end{lstlisting}

Vous devriez voir : \texttt{Barka da zuwa Dabara!}

\subsection{Structure d'un programme Dabara}

Chaque programme Dabara suit cette structure :

\begin{lstlisting}[language=Dabara]
fara
  # Votre code ici
kare
\end{lstlisting}

\begin{description}
    \item[\texttt{fara}] : Marque le début du programme (signifie "commencer")
    \item[\texttt{kare}] : Marque la fin du programme (signifie "terminer")
    \item[Indentation] : Le code est décalé vers la droite pour la lisibilité
\end{description}

\newpage

\section{Les Fondamentaux}

Commençons par les concepts de base de la programmation.

\subsection{Variables et déclaration}

Une variable est comme une boîte qui contient une valeur. Nous utilisons le mot-clé \texttt{naɗa} pour créer des variables.

\begin{lstlisting}[language=Dabara]
fara
  # Déclaration de variables
  naɗa sunan = "Ahmad"
  naɗa shekarun = 25
  naɗa aiki = "Mai shirye-shirye"
  
  # Affichage des variables
  rubuta sunan
  rubuta shekarun
  rubuta aiki
kare
\end{lstlisting}

\subsection{Types de données}

Dabara supporte plusieurs types de données :

\subsubsection{Textes (Chaînes de caractères)}

\begin{lstlisting}[language=Dabara]
fara
  naɗa sunan = "Fatima"
  naɗa salutation = "Sannu Duniya!"
  naɗa adresse = "Niamey, Niger"
  
  rubuta sunan
  rubuta salutation
  rubuta adresse
kare
\end{lstlisting}

\subsubsection{Nombres}

\begin{lstlisting}[language=Dabara]
fara
  naɗa lambar = 42
  naɗa temperature = -5
  naɗa prix = 1250
  
  rubuta lambar
  rubuta temperature
  rubuta prix
kare
\end{lstlisting}

\subsubsection{Booléens (Vrai/Faux)}

\begin{lstlisting}[language=Dabara]
fara
  naɗa gaskiya_ne = gaskiya  # vrai
  naɗa karya_ne = karya      # faux
  
  rubuta gaskiya_ne
  rubuta karya_ne
kare
\end{lstlisting}

\subsection{Affichage avec \texttt{rubuta}}

Le mot-clé \texttt{rubuta} permet d'afficher du texte ou des variables à l'écran.

\begin{lstlisting}[language=Dabara]
fara
  # Afficher du texte simple
  rubuta "Barka da safiya!"
  
  # Afficher une variable
  naɗa sunan = "Musa"
  rubuta sunan
  
  # Afficher plusieurs éléments
  naɗa shekarun = 30
  rubuta "Sunansa: " + sunan
  rubuta "Shekarsa: " + shekarun
kare
\end{lstlisting}

\subsection{Concaténation de textes}

Vous pouvez combiner des textes et des variables avec l'opérateur \texttt{+} :

\begin{lstlisting}[language=Dabara]
fara
  naɗa sunan = "Aisha"
  naɗa gari = "Zinder"
  
  # Concaténation simple
  rubuta "Sunanta: " + sunan
  rubuta "Garinta: " + gari
  
  # Message personnalisé
  rubuta "Sannu " + sunan + " daga " + gari + "!"
kare
\end{lstlisting}

\subsection{Exercices}

\textbf{Exercice 1 : Carte de visite}
Créez un programme qui affiche vos informations personnelles :

\begin{lstlisting}[language=Dabara]
fara
  naɗa sunan = "[Votre nom]"
  naɗa shekarun = [Votre âge]
  naɗa gari = "[Votre ville]"
  naɗa aiki = "[Votre profession]"
  
  rubuta "=================="
  rubuta "KATIN ZIYARA"
  rubuta "=================="
  rubuta "Suna: " + sunan
  rubuta "Shekaru: " + shekarun
  rubuta "Gari: " + gari
  rubuta "Aiki: " + aiki
  rubuta "=================="
kare
\end{lstlisting}

\newpage

\section{Opérations Mathématiques}

Dabara supporte les opérations arithmétiques de base.

\subsection{Les quatre opérations}

\begin{table}[h]
\centering
\begin{tabular}{cll}
\toprule
Opération & Symbole & Exemple \\
\midrule
Addition & + & \texttt{5 + 3 = 8} \\
Soustraction & - & \texttt{10 - 4 = 6} \\
Multiplication & * & \texttt{6 * 7 = 42} \\
Division & / & \texttt{20 / 4 = 5} \\
\bottomrule
\end{tabular}
\caption{Opérateurs arithmétiques}
\end{table}

\subsection{Exemples d'opérations}

\subsubsection{Calculs simples}

\begin{lstlisting}[language=Dabara]
fara
  # Opérations de base
  naɗa jimla = 15 + 25
  naɗa bambanci = 50 - 18
  naɗa ninka = 8 * 7
  naɗa raba = 64 / 8
  
  rubuta "15 + 25 = " + jimla
  rubuta "50 - 18 = " + bambanci
  rubuta "8 * 7 = " + ninka
  rubuta "64 / 8 = " + raba
kare
\end{lstlisting}

\subsubsection{Calculs avec variables}

\begin{lstlisting}[language=Dabara]
fara
  # Calcul de budget
  naɗa kudin_littafi = 500
  naɗa kudin_alkalam = 150
  naɗa kudin_jaka = 1200
  
  naɗa jimlar_kudi = kudin_littafi + kudin_alkalam + kudin_jaka
  
  rubuta "Kudin littafi: " + kudin_littafi
  rubuta "Kudin alkalam: " + kudin_alkalam
  rubuta "Kudin jaka: " + kudin_jaka
  rubuta "Jimlar kudi: " + jimlar_kudi
kare
\end{lstlisting}

\subsection{Priorité des opérations}

Comme en mathématiques traditionnelles, la multiplication et la division sont prioritaires sur l'addition et la soustraction :

\begin{lstlisting}[language=Dabara]
fara
  # Sans parenthèses : multiplication d'abord
  naɗa a = 10 + 5 * 2  # = 20 (pas 30)
  rubuta a
  
  # Avec parenthèses : changement de priorité
  naɗa b = (10 + 5) * 2  # = 30
  rubuta b
kare
\end{lstlisting}

\subsection{Applications pratiques}

\subsubsection{Calculateur de moyenne}

\begin{lstlisting}[language=Dabara]
fara
  # Notes de trois examens
  naɗa daraja1 = 85
  naɗa daraja2 = 92
  naɗa daraja3 = 78
  
  # Calcul de la moyenne
  naɗa jimla = daraja1 + daraja2 + daraja3
  naɗa adadin = 3
  naɗa tsaka_tsaki = jimla / adadin
  
  rubuta "Darajomi:"
  rubuta "Daraja 1: " + daraja1
  rubuta "Daraja 2: " + daraja2
  rubuta "Daraja 3: " + daraja3
  rubuta "Tsaka-tsaki: " + tsaka_tsaki
kare
\end{lstlisting}

\subsubsection{Convertisseur de devises}

\begin{lstlisting}[language=Dabara]
fara
  # Conversion Franc CFA → Euro (taux approximatif)
  naɗa kudi_cfa = 50000
  naɗa mai_canji = 0.0015  # 1 CFA = 0.0015 €
  
  naɗa kudi_euro = kudi_cfa * mai_canji
  
  rubuta kudi_cfa + " CFA = " + kudi_euro + " €"
kare
\end{lstlisting}

\subsection{Gestion des erreurs}

Attention à la division par zéro :

\begin{lstlisting}[language=Dabara]
fara
  naɗa a = 10
  naɗa b = 0
  
  # Ceci générera une erreur
  # naɗa sakamako = a / b
  
  # Vérification avant division
  idan b != 0 {
    naɗa sakamako = a / b
    rubuta sakamako
  } amma {
    rubuta "Ba za a iya raba da sifili ba"
  }
kare
\end{lstlisting}

\newpage

\section{Conditions et Logique}

Les structures conditionnelles permettent à votre programme de prendre des décisions.

\subsection{Structure conditionnelle de base}

\begin{lstlisting}[language=Dabara]
fara
  naɗa shekarun = 18
  
  idan shekarun >= 18 {
    rubuta "Kuna yawa don shiga"
  } amma {
    rubuta "Ba ku yawa a halin yanzu"
  }
kare
\end{lstlisting}

\subsection{Opérateurs de comparaison}

\begin{table}[h]
\centering
\begin{tabular}{cll}
\toprule
Opérateur & Signification & Exemple \\
\midrule
== & Égal à & \texttt{a == b} \\
!= & Différent de & \texttt{a != b} \\
< & Inférieur à & \texttt{a < b} \\
> & Supérieur à & \texttt{a > b} \\
<= & Inférieur ou égal & \texttt{a <= b} \\
>= & Supérieur ou égal & \texttt{a >= b} \\
\bottomrule
\end{tabular}
\caption{Opérateurs de comparaison}
\end{table}

\subsection{Conditions multiples}

\begin{lstlisting}[language=Dabara]
fara
  naɗa daraja = 85
  
  idan daraja >= 90 {
    rubuta "Grade A - Kyau sosai!"
  } ammaina daraja >= 80 {
    rubuta "Grade B - Kyau!"
  } ammaina daraja >= 70 {
    rubuta "Grade C - Yana da kyau"
  } amma {
    rubuta "Grade D - Yana bukata kokarin"
  }
kare
\end{lstlisting}

\subsection{Conditions imbriquées}

\begin{lstlisting}[language=Dabara]
fara
  naɗa shekarun = 25
  naɗa daure = gaskiya
  
  idan shekarun >= 18 {
    idan daure == gaskiya {
      rubuta "Kuna yawa don aure"
    } amma {
      rubuta "Kuna bukata daure"
    }
  } amma {
    rubuta "Ba ku yawa a halin yanzu"
  }
kare
\end{lstlisting}

\subsection{Applications pratiques}

\subsubsection{Calculateur d'IMC}

\begin{lstlisting}[language=Dabara]
fara
  # Données de l'utilisateur
  naɗa takaici = 1.75  # en mètres
  naɗa nauyi = 70      # en kg
  
  # Calcul de l'IMC
  naɗa imc = nauyi / (takaici * takaici)
  
  rubuta "Takaicinku: " + takaici + "m"
  rubuta "Nauyinku: " + nauyi + "kg"
  rubuta "IMC: " + imc
  
  # Interprétation
  idan imc < 18.5 {
    rubuta "Kuna takaici - Ku kara cin abinci"
  } ammaina imc < 25 {
    rubuta "Nauyinku yana da kyau"
  } ammaina imc < 30 {
    rubuta "Kuna samu wani nauyi - Ku yi wasan jiki"
  } amma {
    rubuta "Nauyinku yana da yawa - Ku tattauna likita"
  }
kare
\end{lstlisting}

\subsubsection{Quiz simple}

\begin{lstlisting}[language=Dabara]
fara
  rubuta "Tambaya: Ke ainihin garin Nigeria?"
  rubuta "A) Kano  B) Abuja  C) Lagos"
  naɗa amsa = karɓa()
  
  idan amsa == "B" ama amsa == "b" {
    rubuta "Da kyau! Abuja ne ainihin garin Nigeria."
  } amma {
    rubuta "Amsa ba daidai ba. Abuja ne ainihin garin Nigeria."
  }
kare
\end{lstlisting}

\newpage

\section{Fonctions et Modularité}

Les fonctions permettent de réutiliser du code et d'organiser votre programme.

\subsection{Définition et appel de fonctions}

\begin{lstlisting}[language=Dabara]
fara
  # Définition d'une fonction
  aiki lissafi(a, b) {
    mayar a + b
  }
  
  # Appel de la fonction
  naɗa sakamakon = lissafi(10, 5)
  rubuta "10 + 5 = " + sakamakakon
kare
\end{lstlisting}

\subsection{Paramètres et valeurs de retour}

\begin{lstlisting}[language=Dabara]
fara
  # Fonction avec plusieurs paramètres
  aiki daraja(tsakaci, adadin) {
    mayar tsakaci / adadin
  }
  
  # Fonction qui ne retourne rien
  aiki gaya(sako) {
    rubuta sako
  }
  
  # Utilisation
  naɗa tsaka_tsaki = daraja(100, 4)
  gaya("Tsaka-tsakin lambar shine: " + tsaka_tsaki)
kare
\end{lstlisting}

\subsection{Portée des variables}

\begin{lstlisting}[language=Dabara]
fara
  # Variable globale
  naɗa x = 100
  
  aiki gwaji() {
    # Variable locale
    naɗa x = 999
    mayar x
  }
  
  naɗa sakamako = gwaji()
  
  rubuta "x na waje: " + x          # 100
  rubuta "x na ciki: " + sakamako   # 999
kare
\end{lstlisting}

\subsection{Fonctions récursives}

\begin{lstlisting}[language=Dabara]
fara
  # Calcul du factoriel
  aiki factorial(n) {
    idan n == 0 {
      mayar 1
    } amma {
      mayar n * factorial(n - 1)
    }
  }
  
  naɗa sakamako = factorial(5)
  rubuta "5! = " + sakamako  # 120
kare
\end{lstlisting}

\subsection{Bibliothèque de fonctions utiles}

\begin{lstlisting}[language=Dabara]
fara
  # Fonctions mathématiques
  aiki kashi(daraja, adadin) {
    mayar (daraja * 100) / adadin
  }
  
  aiki cikakkiya(a, b) {
    mayar (a * b) / 100
  }
  
  # Fonctions de texte
  aiki tsawo(jimla) {
    # Implémentation simplifiée
    mayar 0  # À remplacer par comptage réel
  }
  
  # Utilisation
  naɗa adadin_darajomi = 50
  naɗa darajomi_samun = 42
  naɗa kashin_daraja = kashi(darajomi_samun, adadin_darajomi)
  
  rubuta "Kashin daraja: " + kashin_daraja + "%"
kare
\end{lstlisting}

\newpage

\section{Interaction avec l'Utilisateur}

Rendez vos programmes interactifs avec la lecture d'entrées utilisateur.

\subsection{Lecture d'entrées avec \texttt{karɓa}}

\begin{lstlisting}[language=Dabara]
fara
  rubuta "Suna nawa suke?"
  naɗa sunan = karɓa()
  
  rubuta "Shekaru nawa kuke?"
  naɗa shekarun = karɓa()
  
  rubuta "Sannu " + sunan + "! Kuna shekaru " + shekarun
kare
\end{lstlisting}

\subsection{Validation des entrées}

\begin{lstlisting}[language=Dabara]
fara
  rubuta "Shigar da lambar shekarunku:"
  naɗa shekarun = karɓa()
  
  # Conversion et validation
  idan shekarun >= 0 ama shekarun <= 150 {
    rubuta "Shekarunku shine: " + shekarun
  } amma {
    rubuta "Shigarwar ba daidai ba"
  }
kare
\end{lstlisting}

\subsection{Menus interactifs}

\begin{lstlisting}[language=Dabara]
fara
  rubuta "==== MENU ===="
  rubuta "1. Lissafi"
  rubuta "2. Gayyata"
  rubuta "3. Fitowa"
  rubuta "Zabi:"
  
  naɗa zabin = karɓa()
  
  idan zabin == "1" {
    rubuta "Kuna cikin lissafin..."
  } ammaina zabin == "2" {
    rubuta "Kuna cikin gayyatar..."
  } ammaina zabin == "3" {
    rubuta "Mai godiya!"
  } amma {
    rubuta "Zabin ba daidai ba"
  }
kare
\end{lstlisting}

\subsection{Applications pratiques}

\subsubsection{Calculatrice interactive}

\begin{lstlisting}[language=Dabara]
fara
  rubuta "==== CALCULATRICE ===="
  rubuta "Shigar da lambar farko:"
  naɗa a = karɓa()
  
  rubuta "Shigar da lambar biyu:"
  naɗa b = karɓa()
  
  rubuta "Zabi aiki (+, -, *, /):"
  naɗa aiki = karɓa()
  
  idan aiki == "+" {
    naɗa sakamako = a + b
    rubuta "Sakamako: " + sakamako
  } ammaina aiki == "-" {
    naɗa sakamako = a - b
    rubuta "Sakamako: " + sakamako
  } ammaina aiki == "*" {
    naɗa sakamako = a * b
    rubuta "Sakamako: " + sakamako
  } ammaina aiki == "/" {
    idan b != 0 {
      naɗa sakamako = a / b
      rubuta "Sakamako: " + sakamako
    } amma {
      rubuta "Ba za a iya raba da sifili ba"
    }
  } amma {
    rubuta "Aikin ba daidai ba"
  }
kare
\end{lstlisting}

\newpage

\section{Collections et Listes}

Organisez vos données avec des collections.

\subsection{Création de listes}

\begin{lstlisting}[language=Dabara]
fara
  # Liste de textes
  naɗa sunaye = ["Ahmad", "Fatima", "Musa", "Aisha"]
  
  # Liste de nombres
  naɗa lambobi = [10, 20, 30, 40, 50]
  
  # Liste mixte
  naɗa abubuwa = ["Littafi", 25, gaskiya, "Pen"]
  
  # Liste vide
  naɗa faduwa = []
  
  rubuta sunaye
  rubuta lambobi
kare
\end{lstlisting}

\subsection{Accès aux éléments}

\begin{lstlisting}[language=Dabara]
fara
  naɗa sunaye = ["Ahmad", "Fatima", "Musa", "Aisha"]
  
  # Accès par index (à partir de 0)
  rubuta "Na farko: " + sunaye[0]    # Ahmad
  rubuta "Na biyu: " + sunaye[1]     # Fatima
  rubuta "Na ƙarshe: " + sunaye[3]   # Aisha
  
  # Longueur de la liste
  rubuta "Adadin sunaye: " + sunaye.tsawo()
kare
\end{lstlisting}

\subsection{Boucles sur les listes}

\begin{lstlisting}[language=Dabara]
fara
  naɗa sunaye = ["Ahmad", "Fatima", "Musa", "Aisha"]
  
  # Boucle for-each
  don suna a sunaye {
    rubuta "Suna: " + suna
  }
  
  # Boucle avec index
  naɗa i = 0
  don suna a sunaye {
    rubuta "Wanda ke " + i + ": " + suna
    i = i + 1
  }
kare
\end{lstlisting}

\subsection{Opérations sur les listes}

\begin{lstlisting}[language=Dabara]
fara
  naɗa lambobi = [10, 20, 30, 40]
  
  # Ajouter un élément
  lambobi.kara(50)
  rubuta lambobi  # [10, 20, 30, 40, 50]
  
  # Supprimer un élément
  lambobi.cire(0)  # Supprime le premier élément
  rubuta lambobi  # [20, 30, 40, 50]
  
  # Trouver un élément
  naɗa akwai = lambobi.kunshe(30)
  rubuta "Akwai 30? " + akwai  # gaskiya
kare
\end{lstlisting}

\subsection{Applications pratiques}

\subsubsection{Gestionnaire de contacts}

\begin{lstlisting}[language=Dabara]
fara
  # Liste de contacts
  naɗa sunaye = ["Ahmad Ibrahim", "Fatima Ali", "Musa Yusuf"]
  naɗa wayoyi = ["0901234567", "0902345678", "0903456789"]
  
  rubuta "==== LAMBAR WAYOYI ===="
  
  # Afficher tous les contacts
  don i a sunaye.tsawo() {
    rubuta (i + 1) + ". " + sunaye[i] + " - " + wayoyi[i]
  }
  
  # Recherche de contact
  rubuta "Bincika suna:"
  naɗa bincike = karɓa()
  
  don i a sunaye.tsawo() {
    idan sunaye[i].kunshe(bincike) {
      rubuta "An samo: " + sunaye[i] + " - " + wayoyi[i]
    }
  }
kare
\end{lstlisting}

\newpage

\section{Projets Pratiques}

Appliquons nos connaissances à des projets concrets.

\subsection{Projet 1 : Gestionnaire de Budget Personnel}

\begin{lstlisting}[language=Dabara]
fara
  # Variables globales
  naɗa kudin_shiga = []
  naɗa kudin_fitowa = []
  naɗa bayanin_shiga = []
  naɗa bayanin_fitowa = []
  
  # Fonction pour ajouter revenu
  aiki kara_shiga(kudi, bayani) {
    kudin_shiga.kara(kudi)
    bayanin_shiga.kara(bayani)
  }
  
  # Fonction pour ajouter dépense
  aiki kara_fitowa(kudi, bayani) {
    kudin_fitowa.kara(kudi)
    bayanin_fitowa.kara(bayani)
  }
  
  # Fonction pour calculer total
  aiki jimla_jerina(jerin) {
    naɗa jimla = 0
    don kudi a jerin {
      jimla = jimla + kudi
    }
    mayar jimla
  }
  
  # Programme principal
  rubuta "==== MAJEFIN KUDI ===="
  
  # Ajout de revenus
  kara_shiga(50000, "Mai")
  kara_shiga(25000, "Kasuwanci")
  
  # Ajout de dépenses
  kara_fitowa(15000, "Abinci")
  kara_fitowa(8000, "Transport")
  kara_fitowa(5000, "Wasanni")
  
  # Calculs
  naɗa jimlar_shiga = jimla_jerina(kudin_shiga)
  naɗa jimlar_fitowa = jimla_jerina(kudin_fitowa)
  naɗa sauƙin_kudi = jimlar_shiga - jimlar_fitowa
  
  # Affichage des résultats
  rubuta "Jimlar shiga: " + jimlar_shiga
  rubuta "Jimlar fitowa: " + jimlar_fitowa
  rubuta "Sauƙin kudi: " + sauƙin_kudi
  
  # Analyse
  idan sauƙin_kudi > 0 {
    rubuta "Masha Allah! Kuna samun sauƙin kudi."
  } amma {
    rubuta "Yi hankali! Kuna rashin kudi."
  }
kare
\end{lstlisting}

\subsection{Projet 2 : Quiz Multilingue}

\begin{lstlisting}[language=Dabara]
fara
  # Questions et réponses
  naɗa tambayoyi = [
    "Ke ainihin garin Nigeria?",
    "Ƙungiyar wasan kwallon kafa ta Nigeria ta fara a shekara?",
    "Yaushe aka samar da al'umma ta karo miliyan daya a Nigeria?"
  ]
  
  naɗa amsoshi = [
    ["A) Kano", "B) Abuja", "C) Lagos"],
    ["A) 1945", "B) 1960", "C) 1988"],
    ["A) 1960", "B) 1980", "C) 2000"]
  ]
  
idai_daidai = ["B", "A", "C"]
  
  naɗa lambar_daraja = 0
  
  # Fonction pour poser une question
  aiki tambaya(namba) {
    rubuta (namba + 1) + ". " + tambayoyi[namba]
    
    # Afficher les options
    don i a 3 {
      rubuta amsoshi[namba][i]
    }
    
    rubuta "Amsa:"
    naɗa amsa = karɓa()
    
    # Vérifier la réponse
    idan amsa == amsoshi_daidai[namba] {
      rubuta "✓ Da kyau!"
      lambar_daraja = lambar_daraja + 1
    } amma {
      rubuta "✗ Ba daidai ba. Amsa daidai shine: " + amsoshi_daidai[namba]
    }
    
    rubuta "--------------------"
  }
  
  # Programme principal
  rubuta "==== TAMBOYIN TARAYYA ===="
  rubuta "Ƙara ilimi game da tarayya ta Nigeria"
  rubuta "================================"
  
  # Poser toutes les questions
  tambaya(0)
  tambaya(1)
  tambaya(2)
  
  # Afficher le résultat final
  rubuta "==== SAKAMAKO ===="
  rubuta "Darajokanku: " + lambar_daraja + "/3"
  
  idan lambar_daraja == 3 {
    rubuta "Kyau sosai! Kuna da ilmi mai zurfi!"
  } ammaina lambar_daraja == 2 {
    rubuta "Kyau! Kuna da ilmi mai kyau."
  } ammaina lambar_daraja == 1 {
    rubuta "Yana da kyau. Kokarin kara koyo."
  } amma {
    rubuta "Yi kokari. Karatu yana da mahimmanci."
  }
  
  rubuta "Mai godiya da shirin!"
kare

\end{lstlisting}

\newpage

\section{Commandes Utiles et Références}

\subsection{Référence rapide des mots-clés}

\begin{table}[h]
\centering
\begin{tabular}{lll}
\toprule
\textbf{Mot-clé} & \textbf{Signification} & \textbf{Utilisation} \\
\midrule
\texttt{fara} & Commencer & Début de programme \\
\texttt{kare} & Terminer & Fin de programme \\
\texttt{naɗa} & Créer & Déclaration de variable \\
\texttt{rubuta} & Écrire & Affichage \\
\texttt{karɓa} & Recevoir & Lecture d'entrée \\
\texttt{idan} & Si & Condition \\
\texttt{amma} & Sinon & Alternative \\
\texttt{ammaina} & Sinon si & Condition multiple \\
\texttt{aiki} & Travail & Définition de fonction \\
\texttt{mayar} & Retourner & Valeur de retour \\
\texttt{don} & Pour & Boucle \\
\texttt{gaskiya} & Vrai & Booléen positif \\
\texttt{karya} & Faux & Booléen négatif \\
\bottomrule
\end{tabular}
\caption{Mots-clés Dabara}
\end{table}

\subsection{Opérateurs}

\begin{table}[h]
\centering
\begin{tabular}{ll}
\toprule
\textbf{Opérateur} & \textbf{Utilisation} \\
\midrule
\texttt{+} & Addition, concaténation \\
\texttt{-} & Soustraction \\
\texttt{*} & Multiplication \\
\texttt{/} & Division \\
\texttt{=} & Affectation \\
\texttt{==} & Égalité \\
\texttt{!=} & Différence \\
\texttt{<} & Inférieur \\
\texttt{>} & Supérieur \\
\texttt{<=} & Inférieur ou égal \\
\texttt{>=} & Supérieur ou égal \\
\bottomrule
\end{tabular}
\caption{Opérateurs Dabara}
\end{table}

\subsection{Commandes du terminal}

\begin{lstlisting}[language=bash]
# Exécuter un programme
$ dabara nom_du_fichier.ha

# Mode debug (affiche tokens et AST)
$ export DABARA_DEBUG=1 && dabara programme.ha

# Tester tous les exemples
$ ./test_examples.sh

# Compiler Dabara
$ cargo build --release

# Installer localement
$ cargo install --path .
\end{lstlisting}

\subsection{Structure d'un programme complet}

\begin{lstlisting}[language=Dabara]
fara
  # Zone d'importation (futures versions)
  # naɗa std = karɓa("std")
  
  # Déclarations de variables globales
  naɗa version = "1.0"
  naɗa mai_ƙirƙira = "Votre nom"
  
  # Définition des fonctions
  aiki tsawo(jimla) {
    # Implémentation
    mayar 0
  }
  
  # Programme principal
  rubuta "Dabara Version " + version
  rubuta "Ƙirƙirar: " + mai_ƙirƙira
  
  # Votre code ici...
  
kare
\end{lstlisting}

\newpage

\section{Conclusion et Perspectives}

\subsection*{Félicitations !}

Vous venez de terminer votre initiation à la programmation avec Dabara. 🎉

\subsection*{Ce que vous avez appris}

\begin{itemize}
    \item Installation et configuration de l'environnement
    \item Variables, types de données et affichage
    \item Opérations mathématiques
    \item Structures conditionnelles
    \item Fonctions et modularité
    \item Interaction avec l'utilisateur
    \item Collections et listes
    \item Développement de projets complets
\end{itemize}

\subsection*{Prochaines étapes}

\subsubsection{Approfondissement}

\begin{itemize}
    \item Explorer les exemples dans le dossier \texttt{examples/}
    \item Consulter la documentation complète
    \item Participer à la communauté Dabara
    \item Contribuer au développement du langage
\end{itemize}

\subsubsection{Projets à réaliser}

\begin{itemize}
    \item Calculatrice scientifique
    \item Gestionnaire de tâches
    \item Jeu de devinettes
    \item Application de gestion de contacts
    \item Système de vote électronique
\end{itemize}

\subsubsection{Apprentissage avancé}

\begin{itemize}
    \item Algorithmes et structures de données
    \item Programmation orientée objet
    \item Gestion de fichiers
    \item Interfaces graphiques
    \item Déploiement d'applications
\end{itemize}

\subsection*{Ressources}

\subsubsection{Documentation officielle}

\begin{itemize}
    \item \href{https://github.com/votre-compte/dabara}{GitHub Repository}
    \item \href{https://dabara-lang.org/docs}{Documentation officielle}
    \item \href{https://discord.gg/dabara}{Serveur Discord}
\end{itemize}

\subsubsection{Communauté}

\begin{itemize}
    \item Partagez vos créations sur GitHub
    \item Posez des questions dans les Issues
    \item Contribuez aux traductions
    \item Aidez les nouveaux venus
\end{itemize}

\subsection*{Message final}

La programmation est un voyage, pas une destination. Chaque ligne de code que vous écrivez vous rend plus compétent. Continuez à explorer, à expérimenter et à créer.

\vspace{1cm}

\begin{center}
\textbf{Barka da zuwa masana kimiyya!} \\
\textit{Bienvenue dans le monde merveilleux de la science informatique !}
\end{center}

\vspace{2cm}

\begin{flushright}
\textit{Ce guide a été créé avec passion} \\
\textit{pour rendre la programmation accessible à tous} \\
\textit{dans leur langue maternelle.}
\end{flushright}

\newpage

\section*{Annexe : Solutions des exercices}

\subsection*{Solution de l'exercice 1 (Carte de visite)}

\begin{lstlisting}[language=Dabara]
fara
  naɗa sunan = "Ahmad Ibrahim"
  naɗa shekarun = 28
  naɗa gari = "Kano"
  naɗa aiki = "Mai shirye-shirye"
  naɗa email = "ahmad@example.com"
  
  rubuta "=================="
  rubuta "KATIN ZIYARA"
  rubuta "=================="
  rubuta "Suna: " + sunan
  rubuta "Shekaru: " + shekarun
  rubuta "Gari: " + gari
  rubuta "Aiki: " + aiki
  rubuta "Email: " + email
  rubuta "=================="
kare
\end{lstlisting}

\subsection*{Solution bonus : Programme interactif}

\begin{lstlisting}[language=Dabara]
fara
  rubuta "==== SHIRIN BAYANI ===="
  
  rubuta "Suna nawa suke?"
  naɗa sunan = karɓa()
  
  rubuta "Shekaru nawa kuke?"
  naɗa shekarun = karɓa()
  
  rubuta "A wane gari kuke zaune?"
  naɗa gari = karɓa()
  
  rubuta "Me aiki kuke yi?"
  naɗa aiki = karɓa()
  
  # Affichage de la carte
  rubuta ""
  rubuta "=================="
  rubuta "BAYANANKU"
  rubuta "=================="
  rubuta "Suna: " + sunan
  rubuta "Shekaru: " + shekarun
  rubuta "Gari: " + gari
  rubuta "Aiki: " + aiki
  rubuta "=================="
  rubuta "Mai godiya da shirin!"
kare
\end{lstlisting}

\end{document}